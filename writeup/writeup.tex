\documentclass[12pt]{article}
\usepackage{fullpage}
\usepackage{amsmath, amssymb}
\usepackage{setspace}
\usepackage{graphicx}
\usepackage{enumerate}
\usepackage{times}

\newcommand{\eps}{\epsilon}
\newcommand{\kap}{\kappa}
\newcommand{\lam}{\lambda}
\newcommand{\del}{\nabla}

\renewcommand{\vec}[1]{\mathbf{#1}}
\newcommand{\mat}[1]{\mathbf{#1}}

\newcommand{\pp}[2]{\frac{\partial #1}{\partial #2}}

\title{Visualization of Angular Momentum Equilibration during Halo Mergers}
\author{Ryan Brewster \\ Peishi Jiang}

\begin{document}

\maketitle

\section{Introduction}

The energy and matter distribution in the modern universe displays a lot of
very complex structure. It is not well understood exactly how that distribution
evolved. The use of cosmological simulations can help shed light on the
formation of galaxies and other structures. In this context, large collections
of mass are called halos. The main goal of these simulations is to investigate
the formation of small halos and the process by which they merge together to
form the large, complex web of halos observed (or inferred) in the physical
universe.

The Dark Sky simulations  involve a large number of collisionless particles
interacting gravitationally. Though extensive optimizations are necessary due
to the scale of the simulations (over $10^{12}$ particles for the largest data
set), the underlying physical processes are very straightforward.

The computer program used to perform the Dark Sky simulations uses an adaptive
symplectic integrator to propagate the equations of motion for the system. It
also employs several approximations (none of which introduce significant
error), the most significant of which is a spatial tree data structure to make
it possible to approximate gravitational interaction between large collections
of particles.

The initial output of the simulation (henceforth called ``Level 1'' data) is
a series of timestep datasets, each of which contains the list of particle
positions and velocities at that timestep. Analysis of that data was performed,
yielding Level 2 data consisting of a list of halos for each timestep. These
halos were identified using the ROCKSTAR algorithm (which finds clusters of
particles in 6-dimensions space, 3 positional dimensions and 3 velocity
dimensions). ROCKSTAR itself stands for Robust Overdensity Calculation using
K-Space Topologically Adaptive Refinement. This algorithm has been extensively
optimized to provide both very accurate, robust identification of halos and
extremely high performance and parallelization capabilities. Finally, analysis
of the Level 2 data yielded Level 3 data consisting of a merger tree of halos
over time. This merger tree can identify how halos form, merge together, and
separate over time.

Our goal for this visualization project is to find a way to display useful
information during the merging of several small halos into a larger halo. To do
this, we will specify a sequence of timesteps where several halos (at least
two) merge together to form a single halo; this should be possible using the
provided merger tree database. We will then identify all of the particles
involved in this process using the halo database. For each timestep in the
specified time step sequence, we will display every particle involved in the
halo merging. The particles will be colored by a function of their individual
properties (position, velocity) and their collective properties (angular
momentum, proximity from halo center, etc.).

We hypothesize that over the course of time, the distinct halos --- each of
which can be identified by the similar color of its constituent particles ---
will merge together and the particle colors will also quickly equilibrate to
form a homogeneous mix. If this is the case, it should be possible to identify
a time constant for this equilibration process. With the large number of halo
mergers that occur during the Dark Sky simulations, it may even be possible to
find a way to estimate the time constant from other parameters from the small
halos.

\section{Theory}

\section{Results}

\bibliography{writeup.bib}
\bibliographystyle{plain}
\nocite{*}

\end{document}
