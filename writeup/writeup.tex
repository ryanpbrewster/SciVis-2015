\documentclass[12pt]{article}
\usepackage{fullpage}
\usepackage{amsmath, amssymb}
\usepackage{setspace}
\usepackage{graphicx}
\usepackage{enumerate}
\usepackage{times}

\newcommand{\eps}{\epsilon}
\newcommand{\kap}{\kappa}
\newcommand{\lam}{\lambda}
\newcommand{\del}{\nabla}

\renewcommand{\vec}[1]{\mathbf{#1}}
\newcommand{\mat}[1]{\mathbf{#1}}

\newcommand{\pp}[2]{\frac{\partial #1}{\partial #2}}

\title{Visualization of Angular Momentum Equilibration during Halo Mergers}
\author{Ryan Brewster \\ Peishi Jiang}

\begin{document}

\maketitle

\section{Introduction}

The energy and matter distribution in the modern universe displays a lot of
very complex structure. It is not well understood exactly how that distribution
evolved. The use of cosmological simulations can help shed light on the
formation of galaxies and other structures. In this context, large collections
of mass are called halos. The main goal of these simulations is to investigate
the formation of small halos and the process by which they merge together to
form the large, complex web of halos observed (or inferred) in the physical
universe.

The Dark Sky simulations  involve a large number of collisionless particles
interacting gravitationally. Though extensive optimizations are necessary due
to the scale of the simulations (over $10^{12}$ particles for the largest data
set), the underlying physical processes are very straightforward.

The computer program used to perform the Dark Sky simulations uses an adaptive
symplectic integrator to propagate the equations of motion for the system. It
also employs several approximations (none of which introduce significant
error), the most significant of which is a spatial tree data structure to make
it possible to approximate gravitational interaction between large collections
of particles.

The initial output of the simulation (henceforth called ``Level 1'' data) is
a series of timestep datasets, each of which contains the list of particle
positions and velocities at that timestep. Analysis of that data was performed,
yielding Level 2 data consisting of a list of halos for each timestep. These
halos were identified using the ROCKSTAR algorithm (which finds clusters of
particles in 6-dimensions space, 3 positional dimensions and 3 velocity
dimensions). ROCKSTAR itself stands for Robust Overdensity Calculation using
K-Space Topologically Adaptive Refinement. This algorithm has been extensively
optimized to provide both very accurate, robust identification of halos and
extremely high performance and parallelization capabilities. Finally, analysis
of the Level 2 data yielded Level 3 data consisting of a merger tree of halos
over time. This merger tree can identify how halos form, merge together, and
separate over time.

Our goal for this visualization project is to find a way to display useful
information during the merging of several small halos into a larger halo. To do
this, we will specify a sequence of timesteps where several halos (at least
two) merge together to form a single halo; this should be possible using the
provided merger tree database. We will then identify all of the particles
involved in this process using the halo database. For each timestep in the
specified time step sequence, we will display every particle involved in the
halo merging. The particles will be colored by a function of their individual
properties (position, velocity) and their collective properties (angular
momentum, proximity from halo center, etc.).

We hypothesize that over the course of time, the distinct halos --- each of
which can be identified by the similar color of its constituent particles ---
will merge together and the particle colors will also quickly equilibrate to
form a homogeneous mix. If this is the case, it should be possible to identify
a time constant for this equilibration process. With the large number of halo
mergers that occur during the Dark Sky simulations, it may even be possible to
find a way to estimate the time constant from other parameters from the small
halos.


\section{Data Preprocessing}

In this contest, three types of data are provided for the halo identification 
and analysis. The first is the raw particle data, which describes the state (e.g.,
position of the particle, velocity of the particle, acceleration and so forth).
The format of the file is SDF, including two parts (i.e., the layout of the data 
(ASCII header) and the binary data). The second type of the data is Hala catalog,
describing the states of halos at each snapshot time, written in plain text file. 
And the final dataset type is Merger Tree database, including the information of 
merger tree which "links the individual halo catalogs that each represents a snapshot 
in time". 

To identify the halos in each time step and investigate how the particles evolves to 
halos with time, it is crucial to conduct relevant numerical analysis on both raw 
particle datasets and information of halo and merger tree. Therefore, the unit conversion 
among the three data types is necessary. A short summary on the main elements in the 
raw particle datasets and halo catalog & merger tree is provided in Table 1.

\begin{table}[h]
\caption {Units of main elements in the three data types} \label{tab:title} 
\begin{tabular}{|c|c|c|c}
 {\bf element} & {\bf unit} & {\bf element} & {\bf unit}  \\
 length & kpc & Halo position & Mpc/h (comoving)   \\
 velocity & kpc/Gyr & Halo velocity & km/s (physical)  \\
 mass & 1e10 Msun  & Halo angular momenta & (Msun/h) * (Mpc/h) * km/s (physical)  \\
 time & Gyr &  &  
\end{tabular}
\end{table}


\section{Data Analysis}
The objective of this visualization project is to simulate how two halos evolve with 
time and are combined with each other finally. Therefore, we need to tackle and present
the following issues: (1) identifying its belonging particles given a halo at each time 
step; (2) visualizing the magnitude of the angular momentum of each belonging particle 
in a specific halo; (3) animating the combination process of two halos over time;
(4) analyzing the change of particle angular momentum spectrum of two halos over time.

\subsection{The angular momentums of the particles from a specific halo}
The position information of particles (from the raw particle file) and halos (from the 
merger tree file \"hlist_*\") at each time step are utilized to find the belonging particles
in a given halo. Nevertheless, due to the different data generating systems, it is necessary 
to conduct a conversion of both units and coordinates between the raw data file and 
\"hlist_*\" file. A conversion function is thus created to perform the unit scaling and
coordinate shifting as follows:

\begin{equation}
f(\vec{r}) = a \vec{r} + \vec{b}
\end{equation}

where $f$ is the conversion function; $\vec{r}$ is the coordinates of the halos; 
$a$ is the unit conversion coefficient between the raw particle and the halo and 
$\vec{b}$ is the relative position of the halo coordinates with regard to the particle
coordinates

Moreover, we roughly define that the particle which is within a certain value from 
the center of a halo is a belonging particle of that halo. The certain value is chosen as 
the ratio of the halo radius and the scale radius. In addition, the angular momentum of 
a particle in its halo is given by the product of the relative position of the particle
in its halo and the relative linear momentum between the particle and the halo, which 
are expressed as follows

\begin{equation}
\vec{L} = (\vec{r_1} - \vec{r_h}) \times (\vec{p_1} - \vec{p_h}}
\end{equation}

where $\vec{L}$ is the angular momentum of a particle; $\vec{r_1}$ and $\vec{r_h}$ are
the positions of the particle and the halo, respectively and $\vec{p_1}$ and $\vec{p_h}$ 
are the linear momentums of the particle and the halo, respectively.

Finally, VTK is used for the visualization of the particles of a specific halo at each 
time step through volume rendering. The magnitudes of the angular momentums of all the 
particles in the halo are colored to provide a general insight of how the particles move 
inside the halo.

\subsection{The evolving of two combined halos}


\section{Results}

\bibliography{writeup.bib}
\bibliographystyle{plain}
\nocite{*}

\end{document}
